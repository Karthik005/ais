\documentclass[12pt]{article}

\usepackage{enumerate}
\usepackage{algorithm}
\usepackage{algorithmicx}
% \usepackage[noend]{algpseudocode}
\usepackage{amsmath}
\usepackage{amssymb}
\usepackage{fancyhdr}
\usepackage{pstricks}    %for embedding pspicture.
\usepackage{graphicx}
\usepackage{tikz}


%-------------list of macros for this scribe-----------------
 \newcommand{\Sen}{\ensuremath{\bf S}}
 \def\NoNumber#1{{\def\alglinenumber##1{}\State #1}\addtocounter{ALG@line}{-1}}
 \newcommand{\n}{\ensuremath{\mathcal n}}
 \newcommand{\Con}{\ensuremath{\mid\mid}}
 \newcommand{\Com}{\newline \Comment}
  \newcommand{\Rec}{\ensuremath{\bf R}}
  \newcommand{\Adv}{\ensuremath{\mathcal A}}
    \newcommand{\Gen}{\ensuremath{\mathsf Gen}}
       \newcommand{\Enc}{\ensuremath{\mathsf Enc}}
       \newcommand{\Dec}{\ensuremath{\mathsf Dec}}
           \newcommand{\Key}{\ensuremath{\mathsf k}}
  \newcommand{\Message}{\ensuremath{\mathsf m}}
    \newcommand{\Cipher}{\ensuremath{\mathsf c}}
       \newcommand{\MSpace}{\ensuremath{\mathcal M}}
          \newcommand{\KSpace}{\ensuremath{\mathcal K}}
          \newcommand\numberthis{\addtocounter{equation}{1}\tag{\theequation}}
            \newcommand{\CSpace}{\ensuremath{\mathcal C}}



\begin{document}
\begin{algorithm}[H]
\caption{ARTIS\_PP}
\begin{algorithmic}[1]
\State Generate $\{d_1,d_2,.....,d_n\}$.
\For
\newline \Comment {It is important to note that the pair of values ($p_a^0$, $p_a^1$) is random since one is the complement of the other}
\State For the sake of simplicity, let make the following assumptions:
\newline  G=AND, $p_a^0$ = 0, $p_a^1$ = 1, $p_b^0$ = 1, $p_b^1$ = 0, $p_c^0$ = 0, $p_c^1$ = 1
\State $P_1$ then generates the following double encryptions:
\newline $C_{00} = E_{K_a^0}(E_{K_b^1}(K_c^{G(0,1)} || 0))= E_{K_a^0}(E_{K_b^1}(K_c^0 || 0))$
\newline $C_{01} = E_{K_a^0}(E_{K_b^0}(K_c^0 || 0))$
\newline $C_{10} = E_{K_a^1}(E_{K_b^1}(K_c^1 || 1))$
\newline $C_{11} = E_{K_a^1}(E_{K_b^0}(K_c^0 || 0))$
\State $P_1$ sends $C_{mn} \forall m,n \in \{0,1\}$  to $P_2$
\State $P_1$ selects its input $a=1$(say) and sends the corresponding key appended to the corresponding random bit , i.e, $K_a^1 || 1$
\State $P_2$ then selects its input $b=0$(say) and a round of oblivious transfer (OT) is performed with $P_1$ to obtain the corresponding key and random bit $K_b^0 || 1$
\State $P_2$ thus obtains the two random bits $(1,1)$ and proceeds to decrypt the corresponding ciphertext $C_{11}$ to obtain $K_c^0 || 0$
\State If G is not a final gate, the output acts as input to the next gate in the circuit. Else, the answer(y) is output as the circuit result.
\end{algorithmic}
\end{algorithm}

\end{document}
